
\documentclass[12pt,a4paper]{ctexart}%支持中文
	\usepackage[top=2cm, bottom=2cm, left=2cm, right=2cm]{geometry}%调节页边距

	\usepackage{algorithm}
	\usepackage{algorithmicx}
	\usepackage{algpseudocode}%支持伪代码
	
	\usepackage{tikz}
	\usetikzlibrary{arrows,shapes,shadows}%使用tikz工具包画图

	\usepackage{amsmath} 
	\usepackage{amssymb}%编辑数学公式
	
	\usepackage{graphicx}%支持图片插入
	
	\usepackage{multirow}%支持复杂表格
	
\title{Project2 :Static routing forwarding}
\author{刘昭策 2018E8018661085\\
	黄文韬 2018E8018661100}
\date{May , 2019}%作者名、标题名、日期

\CTEXsetup[format={\Large\bfseries}]{section}%CTEX中标题左对齐
\renewcommand\thesubsection{\roman{subsection}}%设置subsection按小写roman字符编号

\setlength{\parindent}{2em}%首行缩进
%\setlength{\parskip}{0.5em}%段落间距
%\setlength{\lineskip}{0.5em}%行间距

\floatname{algorithm}{算法}
\renewcommand{\algorithmicrequire}{\textbf{Input:}}
\renewcommand{\algorithmicensure}{\textbf{Output:}}%输入输出

\begin{document}	
	\maketitle
	\pagestyle{plain}%页码在页下部居中且没有页眉
%---------------------------section 1--------------------------------
	\section*{一、实验目的}
		\begin{itemize}
			\item 理解IP、ARP、ICMP协议的工作机制
			\item 实现IP地址查找与数据包转发
			\item 实现ARP请求和应答、ARP数据包缓存管理
			\item 实现ICMP消息的发送
			\item 学习ping,traceroute等网络命令的基本功能和使用方法
		\end{itemize}
		
	\section*{二、实验环境}
	Ubuntu16.04 + Mininet
	
	\section*{三、实验内容}
		\subsection{IP协议相关}
		主要内容:最长前缀匹配规则
				
		\subsection{ARP协议相关}
		主要内容:IP-Mac地址映射基本概念,ARP请求、ARP回应数据包格式,ARP条目查询
		
		\subsection{ICMP协议相关}
		主要内容:实验中涉及到的几种ICMP数据包格式(路由表查找失败、ARP查询失败、TTL值减为0、收到ping本端口的数据包)
	
	\section*{四、实验流程}
		\setcounter{subsection}{0}
		
		\subsection{处理IP数据包}
		判断是否为ICMP echo请求,否则利用最长前缀匹配规则转发IP包,转发IP包之前需要检查TTL,更新checksum等
		
		\subsection{发送ICMP数据包}
		满足下述四个条件时发送ICMP数据包\\
		1,TTL值为0\\
		2,查找不到路由表条目\\
		3.收到ping本端口的包\\
		4,ARP查询失败
		
		
		\subsection{ARP缓存管理}
		在没有触发前三个条件时,将数据包添加到ARP缓存的等待队列中,并发送ARP请求\\
		注意缓存管理要避免死锁的出现(lock unlock)
		
		\subsection{处理ARP请求和应答}
		收到ARP数据包时,先判断是ARP请求还是ARP应答
		
		处理ARP请求数据包时,先发送ARP回复数据包,再将IP-Mac地址映射对插入ARP缓存中,并查找等待队列中是否有数据包符合地址映射对,如有,则将对应的数据包发送
		
		处理ARP回复数据包时,同理也先将IP-Mac地址映射对插入ARP缓存中,再查找等待队列中是否有数据包符合地址映射对,如有,则将对应的数据包发送
		
		
			
	\section*{五、实验结果与分析}
		\setcounter{subsection}{0}
		\subsection{实验一结果截图与分析}
	
		\subsection{实验二结果截图与分析}
		
\end{document}
